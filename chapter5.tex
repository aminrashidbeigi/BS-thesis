\chapter{جمع‌بندی، نتيجه‌گيري و پیشنهادات}
%%%%%%%%%%%%%%%%%%%%%%%%%%%%%%%%%%%%%%%%%%%

\section{جمع‌بندی و نتیجه‌گیری}
امروزه با گسترش فعالیت‌های محققان در زمینه‌های سخت‌افزاری و نرم‌افزاری، پیشرفت‌هایی حاصل شده است که همکاری هرچه بیشتر این دو موضوع مطالعاتی را هموار می‌کند.

در صنعت‌های مختلف از قطعات سخت‌افزاری برای خودکارسازی فرآیندها و روندها استفاده می‌شود. یکی از محیط‌های با پتانسیل بالا، انبارها و فروشگاه‌های بزرگ و با تعداد محصولات زیاد هستند. در حال حاضر در کشور ما بیشتر فرآیندهای جمع‌آوری کالاها و البته بسته‌بندی به صورت انسانی و بدون کمک چندان از کامپیوتر انجام می‌گیرند. این کار باعث شده است تا مدیریت انبار‌های بزرگ به گلوگاه تبدیل شود. از مهمترین مشکلات مدیریتی در انبارهای بزرگ، تعیین جایگاه محصولات است؛ به طوری که کمترین هزینه را برای مجموعه داشته باشد. اگر محصولات چیدمان بهینه‌ای نداشته باشند،‌ به نیروهای بیشتر و در نتیجه هزینه بیشتری برای جمع‌آوری محصولات نیاز است. 

هدف این پروژه کاهش این هزینه‌ها با فراهم کردن زیرساخت‌هایی جهت مدیریت راحت‌تر عناصر موجود در انبارها است. 

برای پیاده‌سازی پروژه، در انبارهای بزرگ و در بخش‌ها و قفسه‌های مختلف،‌ بورد‌های سخت افزاری‌ای قرار داده می‌شود که بر روی هر بورد، نمایشگری در جهت تعامل کاربر فراهم شده است. مسئولین چیدمان محصولات در انبار پس از تغییر جایگاه محصولات، به کمک برنامه‌ای که بر روی این بوردها نصب شده است، جایگاه فعلی محصولات را نیز بروزرسانی می‌کنند. همچنین پس از جمع‌آوری کالاهای سفارش‌های ثبت شده، کارکنان تعداد کالاهایی که از انبار کم شده است را بروزرسانی می‌کنند. این کار توسط برنامه‌ای که برای جمع‌آورندگان کالاهای سفارش‌های داخل انبار نوشته شده‌است انجام می‌گیرد. در نهایت برنامه‌ای برای مدیران انبار فراهم شده است تا به راحتی بتوانند وضعیت موجود در انبار را مشاهده کنند و در صورت نیاز، اقدام به تغییر عناصر آن بکنند.


\section{کارهای آینده}
مشکل اصلی این پروژه، هزینه نسبتا بالای بورد رزبری‌پای و همچنین نمایشگر لمسی در بازار کنونی است. با توجه به پیاده‌سازیِ تحت وب برنامه‌های سمت کاربر، این پروژه با هر گونه سخت‌افزاری که توانایی جستجو در صفحات وب را داشته باشد انطباق داده می‌شود. همچنین با توجه به این که تمامی عملیات و تغییرات در میزبانی، جدا از سخت‌افزار در تعامل با کاربر انجام می‌شود، سخت‌افزاری با مشخصات نه چندان زیاد هم پاسخگوی نیازمندی‌های پروژه است. یکی از راه‌های ارزان‌تر برای عملی ساختن پروژه، استفاده از رایانه‌های لوحی\LTRfootnote{\lr{Tablet}} ارزان قیمت و البته با سخت‌افزار نسبتا ضعیف است.

از دیگر موارد مفید، می‌توان به استفاده از سنسورها و عملگرهای مختلف اشاره کرد. سنسور دما برای خواندن دمای قفسه در انبارهای مواد غذایی و یا محصولات حساس به دمای محیط، سنسور تشخیص دود و یا آتش‌سوزی، سنسور تشخیص میزان رطوبت هوا برای محصولات الکتریکی و دیگر محصولاتی که به رطوبت اطراف حساس هستند، می‌توانند در بهبود  کارایی این سخت‌افزارها مفید واقع شوند. همچنین به صورت نرم‌افزاری می‌توان اخطارهایی در سیستم دیده‌بان\LTRfootnote{\lr{Monitoring}} تعریف کرد تا در صورت بروز مشکل، از طریق سرویس‌هایی مانند پیامک به مسئول مربوطه اطلاع داده شود.

استفاده از رباتیک برای انبارهای انبوه و بزرگ، چشم‌انداز غایی این پروژه است. ربات‌ها می‌توانند در جمع‌آوری محصولات داخل انبار و فروشگاه‌ها و همچنین چیدن محصولات در بخش‌های مختلف جایگزین انسان شوند. از بارزترین نمونه‌های استفاده از رباتیک در فرایندهای انبارداری، شرکت آمازون است. این شرکت هرسال مسابقاتی را جهت هوشمندسازی انبارداری در زمینه‌های چیدمان و جمع‌آوری محصولات برگزار می‌کند. درسال‌های اخیر در شرکت آمازون ربات‌ها وظیفه جمع‌آوری و همچنین چیدن محصولات را بر عهده دارند\cite{Amazon}.

حوزه هوش مصنوعی و علم داده را نیز می‌توان در این پروژه دخیل کرد. به مرور زمان داده‌های ارزشمندی از جایگاه محصولات و میزان مصرف محصولات در بخش‌های مختلف به دست می‌آید که به مدیر و مسئول چیدمان انبار کمک شایانی خواهد کرد. درنهایت از این داده‌ها می‌توان در پیدا کردن مسیرهای بهینه جمع‌آوری محصولات استفاده کرد.