\chapter{تعریف مسئله، راه حل پیشنهادی و تعریف نیازمندی‌ها}
\section{مقدمه}
در این قسمت ابتدا به تعریف و طرح مسئله پرداخته می‌شود. سپس راه‌حلی پیشنهاد داده می‌شود و در ادامه نیازمندی‌های پروژه بر اساس راه حل پیشنهادی تعریف می‌شود.
پیش از آغاز پیاده‌سازی نرم‌افزار لازم است درک جامع و کاملی از مسئله پیدا کنیم و فهرست دقیقی از نیازمندی‌های آن تهیه نماییم. در این فصل سعی شده است پس از بیان مسئله، نیازمندی‌ها و روش حل آن مشخص شوند.

\section{تعریف مسئله}
<<<<<<< HEAD
مدیریت و یافتن کالاها در فروشگاه‌ها و انبار‌های بزرگ، یکی از دغدغه‌های اصلی فروشندگان و انباردارهاست. هنگامی که سفارشی ثبت می‌شود، جمع‌آورندگان محصولات در انبارها وظیفه دارند در انبوهی از کالاها و بخش‌های انبار، کالاهای ثبت‌شده در سفارش را جمع‌آوری کنند. اگر انبار مربوطه تعداد زیادی کالا داشته باشد و یا انباری مربوط به کسب‌و‌کارهایی با محصولات پرمصرف باشد، این موضوع به گلوگاه تبدیل می‌شود. همچنین مشتریان فروشگاه‌های زنجیره‌ای همواره با پیدا کردن لیست محصولات مورد نیاز خود در بین انبوهی از کالاها مشکل دارند. کاهش زمان صرف‌شده برای جمع‌آوری کالاها، برای مدیران 
=======
مدیریت و یافتن کالاها در فروشگاه‌ها و انبار‌های بزرگ، یکی از دغدغه‌های اصلی فروشندگان و انباردارهاست. هنگامی که سفارشی ثبت می‌شود، جمع‌آورنگان محصولات در انبارها وظیفه دارند در انبوهی از کالاها و بخش‌های انبار، کالاهای ثبت‌شده در سفارش را جمع‌آوری کنند. اگر انبار مربوطه تعداد زیادی کالا داشته باشد و یا انباری مربوط به کسب‌و‌کارهایی با محصولات پرمصرف باشد، این موضوع به گلوگاه تبدیل می‌شود. همچنین مشتریان فروشگاه‌های زنجیره‌ای همواره با پیدا کردن لیست محصولات مورد نیاز خود در بین انبوهی از کالاها مشکل دارند. کاهش زمان صرف‌شده برای جمع‌آوری کالاها، برای مدیران 
>>>>>>> 0b914906bc0a1f3ca7b01ffa78967759bf4780dc
انبارها، از اهمیت ویژه‌ای برخوردار است.


\section{راه حل پیشنهادی} 
راه حل پیشنهادی برای حل این مسئله، طراحی و پیاده‌سازی سامانه‌ای جهت مدیریت، ایجاد تغییرات و کنترل کالاها در جایگاه‌های متفاوت در این محیط‌هاست. این سامانه متشکل از قطعه‌های سخت‌افزاری و همچنین نرم‌افزار لازم برای تعامل با کاربر است. این سامانه از طریق بوردها و صفحه‌های نمایش لمسی قرار گرفته بر روی آن، در قسمت‌ها و دسته‌های مختلف انبار و فروشگاه، این اجازه را به کارکنان انبارها و فروشگاه‌ها می‌دهد که موجودی کالا‌های خود را به روز‌رسانی کنند و از طریق نرم‌افزار مربوطه، مکان، موجودی و سایر مشخصات کالا را جست‌و‌جو کنند. همچنین به یک برنامه سمت کاربر جهت مدیریت داده‌ها و کاربران نیاز است.


\section{نیازمندی‌های پروژه}
با توجه به موارد مطرح شده در قسمت تعریف مسئله و همچنین در قسمت راه حل پیشنهادی، نیازمندی‌هایی به وجود آمد که در ادامه به آن‌ها پرداخته می‌شود.

\subsection{رابط کاربری مدیریت}
پروژه شامل کاربرها و محصولات، بخش‌ها\LTRfootnote{\lr{Segments}}، سفارش‌ها و دسته‌بندی‌های مختلف است. برای مدیریت و کنترل این قسمت‌ها، به یک محیطی نیاز است که مدیر، دسترسی تغییر بخش‌های مختلف را بر اساس نیاز داشته باشد.

مدیر در محیط مدیریت\LTRfootnote{\lr{Admin Panel}} باید بتواند اعمال زیر را انجام دهد:
\begin{itemize}
   	\item اضافه کردن، ویرایش و یا حذف یک محصول
   	\item اضافه کردن، ویرایش و یا حذف یک دسته‌بندی
   	\item اضافه کردن، ویرایش و یا حذف یک بخش
   	\item مشاهده و ویرایش سفارش‌های ثبت شده
\end{itemize}
با توجه به این که ممکن است بر اساس نیاز، لازم شود افراد دیگری غیر از مدیر اصلی توانایی تغییر بخش‌های ذکر شده را داشته باشند، باید امکانِ دادن دسترسی به کاربران دیگر نیز داده شود. همچنین نیاز است سطح دسترسی کاربران تحت عنوان نقش‌های مختلف نیز تعریف شوند. از وظایف مدیر اصلی رابط کاربری مدیریت، تعیین دسترسی‌های کاربران و نقش‌های مختلف برنامه است.

\subsection{برنامه جمع‌آوری محصولات سفارش‌ها}
جمع‌آورندگان\LTRfootnote{\lr{Picker}} داخل انبار و فروشگاه نیاز دارند تا از سفارش‌های ثبت شده مطلع شوند تا به جمع‌آوری محصولات آن سفارش بپردازند. همچنین پس از جمع‌آوری محصولات، به مدیر و بقیه جمع‌آورندگان باید اطلاع دهند که سفارش جمع‌آوری گردید.

این برنامه شامل موارد زیر است:
\begin{itemize}
	\item مشاهده سفارش‌های ثبت شده و وضعیت آن‌ها
	\item مشاهده محصولات سفارش
	\item اطلاع‌ دادن پس از جمع‌آوری محصولات سفارش
\end{itemize}

\subsection{برنامه انتخاب محصولات بخش‌ها‌}
بخش‌های مختلف در انبارها و فروشگاه‌ها دارای محصولات متفاوتی هستند. این محصولات در طول زمان اضافه، کم ویا جابجا می‌شوند.

<<<<<<< HEAD
در این پروژه به قطعات سخت‌افزاری جهت قرارگیری بر روی بخش‌های مختلف نیاز است تا کاربر بتواند به راحتی با آن تعامل برقرار کند و محصولات فعلی بخش  و موجودی آن‌ها را ثبت کنند. برای ثبت محصولات، علاوه بر قطعات سخت‌افزاری، به برنامه‌ای نیاز است تا محصولات موجود را نمایش دهد و پس از تغییر جایگاه محصولات، آن تغییر را در پایگاه‌داده ثبت کند. این برنامه بر روی یک نمایشگر لمسی به مسئول جمع‌آوری نمایش داده می‌شود تا آن فرد در کمترین زمان تغییرات جدید را بروزرسانی کند. این نمایشگرها بر روی قفسه‌های انبار نصب می‌شود.

=======
در این پروژه به قطعات سخت‌افزاری جهت قرارگیری بر روی بخش‌های مختلف نیاز است تا کاربر بتواند به راحتی با آن تعامل برقرار کند و محصولات فعلی بخش  و موجودی آن‌ها را ثبت کنند. برای ثبت محصولات، علاوه بر قطعات سخت‌افزاری، به برنامه‌ای نیاز است تا محصولات موجود را نمایش دهد و پس از تغییر جایگاه محصولات، آن تغییر را در پایگاه‌داده ثبت کند.
>>>>>>> 0b914906bc0a1f3ca7b01ffa78967759bf4780dc

\section{نتیجه‌گیری}
در این قسمت مسئله مورد نظر تعریف و شرح داده شد. همچنین راه‌حلی پیشنهادی برای مسئله ارائه شد. در نهایت و با توجه به راه حل، نیازمندی‌های مسئله تعریف شد. این نیازمندی‌ها شامل برنامه‌های سمت کاربر و همچنین مدیریت مجموعه است.
