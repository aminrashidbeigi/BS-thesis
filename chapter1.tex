\chapter{راهنمای استفاده از الگوی لاتک دانشگاه صنعتی امیرکبیر(پلی‌تکنیک تهران)}

\section{مقدمه}
حروف‌چینی پروژه کارشناسی، پایان‌نامه یا رساله یکی از موارد پرکاربرد استفاده از زی‌پرشین است. از طرفی، یک پروژه، پایان‌نامه یا رساله،  احتیاج به تنظیمات زیادی از نظر صفحه‌آرایی  دارد که ممکن است برای
یک کاربر مبتدی، مشکل باشد. به همین خاطر، برای راحتی کار کاربر، یک کلاس با نام 
\verb;AUTthesis;
 برای حروف‌چینی پروژه‌ها، پایان‌نامه‌ها و رساله‌های دانشگاه صنعتی امیرکبیر با استفاده از نرم‌افزار زی‌پرشین،  آماده شده است. این فایل به 
گونه‌ای طراحی شده است که کلیه خواسته‌های مورد نیاز  مدیریت تحصیلات تکمیلی دانشگاه صنعتی امیرکبیر را برآورده می‌کند و نیز، حروف‌چینی بسیاری
از قسمت‌های آن، به طور خودکار انجام می‌شود.

کلیه فایل‌های لازم برای حروف‌چینی با کلاس گفته شده، داخل پوشه‌ای به نام
\verb;AUTthesis;
  قرار داده شده است. توجه داشته باشید که برای استفاده از این کلاس باید فونت‌های
  \verb;Nazanin B;،
 \verb;PGaramond;
 و
  \verb;IranNastaliq;
    روی سیستم شما نصب شده باشد.
\section{این همه فایل؟!}\label{sec2}
از آنجایی که یک پایان‌نامه یا رساله، یک نوشته بلند محسوب می‌شود، لذا اگر همه تنظیمات و مطالب پایان‌نامه را داخل یک فایل قرار بدهیم، باعث شلوغی
و سردرگمی می‌شود. به همین خاطر، قسمت‌های مختلف پایان‌نامه یا رساله  داخل فایل‌های جداگانه قرار گرفته است. مثلاً تنظیمات پایه‌ای کلاس، داخل فایل
\verb;AUTthesis.cls;، 
تنظیمات قابل تغییر توسط کاربر، داخل 
\verb;commands.tex;،
قسمت مشخصات فارسی پایان‌نامه، داخل 
\verb;fa_title.tex;,
مطالب فصل اول، داخل 
\verb;chapter1;
و ... قرار داده شده است. نکته مهمی که در اینجا وجود دارد این است که از بین این  فایل‌ها، فقط فایل 
\verb;AUTthesis.tex;
قابل اجرا است. یعنی بعد از تغییر فایل‌های دیگر، برای دیدن نتیجه تغییرات، باید این فایل را اجرا کرد. بقیه فایل‌ها به این فایل، کمک می‌کنند تا بتوانیم خروجی کار را ببینیم. اگر به فایل 
\verb;AUTthesis.tex;
دقت کنید، متوجه می‌شوید که قسمت‌های مختلف پایان‌نامه، توسط دستورهایی مانند 
\verb;input;
و
\verb;include;
به فایل اصلی، یعنی 
\verb;AUTthesis.tex;
معرفی شده‌اند. بنابراین، فایلی که همیشه با آن سروکار داریم، فایل 
\verb;AUTthesis.tex;
است.
در این فایل، فرض شده است که پایان‌نامه یا رساله شما، از5 فصل و یک پیوست، تشکیل شده است. با این حال، اگر
  پایان‌نامه یا رساله شما، بیشتر از 5 فصل و یک پیوست است، باید خودتان فصل‌های بیشتر را به این فایل، اضافه کنید. این کار، بسیار ساده است. فرض کنید بخواهید یک فصل دیگر هم به پایان‌نامه، اضافه کنید. برای این کار، کافی است یک فایل با نام 
\verb;chapter6;
و با پسوند 
\verb;.tex;
بسازید و آن را داخل پوشه 
\verb;AUTthesis;
قرار دهید و سپس این فایل را با دستور 
\texttt{\textbackslash include\{chapter6\}}
داخل فایل
\verb;AUTthesis.tex;
و بعد از دستور
\texttt{\textbackslash include\{chapter6\}}
 قرار دهید.

\section{از کجا شروع کنم؟}
قبل از هر چیز، بدیهی است که باید یک توزیع تِک مناسب مانند 
\verb;Live TeX;
و یک ویرایش‌گر تِک مانند
\verb;Texmaker;
را روی سیستم خود نصب کنید.  نسخه بهینه شده 
\verb;Texmaker;
را می‌توانید  از سایت 
 \href{http://www.parsilatex.com}{پارسی‌لاتک}%
\LTRfootnote{\url{http://www.parsilatex.com}}
 و
\verb;Live TeX;
را هم می‌توانید از 
 \href{http://www.tug.org/texlive}{سایت رسمی آن}%
\LTRfootnote{\url{http://www.tug.org/texlive}}
 دانلود کنید.
 
در مرحله بعد، سعی کنید که  یک پشتیبان از پوشه 
\verb;AUTthesis;
 بگیرید و آن را در یک جایی از هارددیسک سیستم خود ذخیره کنید تا در صورت خراب کردن فایل‌هایی که در حال حاضر، با آن‌ها کار می‌کنید، همه چیز را از 
 دست ندهید.
 
 حال اگر نوشتن پایان‌نامه اولین تجربه شما از کار با لاتک است، توصیه می‌شود که یک‌بار به طور سرسری، کتاب «%
\href{http://www.tug.ctan.org/tex-archive/info/lshort/persian/lshort.pdf}{مقدمه‌ای نه چندان کوتاه بر
\lr{\LaTeXe}}\LTRfootnote{\url{http://www.tug.ctan.org/tex-archive/info/lshort/persian/lshort.pdf}}»
   ترجمه دکتر مهدی امیدعلی، عضو هیات علمی دانشگاه شاهد را مطالعه کنید. این کتاب، کتاب بسیار کاملی است که خیلی از نیازهای شما در ارتباط با حروف‌چینی را برطرف می‌کند.
 
 
بعد از موارد گفته شده، فایل 
\verb;AUTthesis.tex;
و
\verb;fa_title;
را باز کنید و مشخصات پایان‌نامه خود مثل نام، نام خانوادگی، عنوان پایان‌نامه و ... را جایگزین مشخصات موجود در فایل
\verb;fa_title;
 کنید. دقت داشته باشید که نیازی نیست 
نگران چینش این مشخصات در فایل پی‌دی‌اف خروجی باشید. فایل 
\verb;AUTthesis.cls;
همه این کارها را به طور خودکار برای شما انجام می‌دهد. در ضمن، موقع تغییر دادن دستورهای داخل فایل
\verb;fa_title;
 کاملاً دقت کنید. این دستورها، خیلی حساس هستند و ممکن است با یک تغییر کوچک، موقع اجرا، خطا بگیرید. برای دیدن خروجی کار، فایل 
\verb;fa_title;
 را 
\verb;Save;، 
(نه 
\verb;As Save;)
کنید و بعد به فایل 
\verb;AUTthesis.tex;
برگشته و آن را اجرا کنید. حال اگر می‌خواهید مشخصات انگلیسی پایان‌نامه را هم عوض کنید، فایل 
\verb;en_title;
را باز کنید و مشخصات داخل آن را تغییر دهید.%
\RTLfootnote{
برای نوشتن پروژه کارشناسی، نیازی به وارد کردن مشخصات انگلیسی پروژه نیست. بنابراین، این مشخصات، به طور خودکار،
نادیده گرفته می‌شود.
}
 در اینجا هم برای دیدن خروجی، باید این فایل را 
\verb;Save;
کرده و بعد به فایل 
\verb;AUTthesis.tex;
برگشته و آن را اجرا کرد.

برای راحتی بیشتر، 
فایل 
\verb;AUTthesis.cls;
طوری طراحی شده است که کافی است فقط  یک‌بار مشخصات پایان‌نامه  را وارد کنید. هر جای دیگر که لازم به درج این مشخصات باشد، این مشخصات به طور خودکار درج می‌شود. با این حال، اگر مایل بودید، می‌توانید تنظیمات موجود را تغییر دهید. توجه داشته باشید که اگر کاربر مبتدی هستید و یا با ساختار فایل‌های  
\verb;cls;
 آشنایی ندارید، به هیچ وجه به این فایل، یعنی فایل 
\verb;AUTthesis.cls;
دست نزنید.

نکته دیگری که باید به آن توجه کنید این است که در فایل 
\verb;AUTthesis.cls;،
سه گزینه به نام‌های
\verb;bsc;,
\verb;msc;
و
\verb;phd;
برای تایپ پروژه، پایان‌نامه و رساله،
طراحی شده است. بنابراین اگر قصد تایپ پروژه کارشناسی، پایان‌نامه یا رساله را دارید، 
 در فایل 
\verb;AUTthesis.tex;
باید به ترتیب از گزینه‌های
\verb;bsc;،
\verb;msc;
و
\verb;phd;
استفاده کنید. با انتخاب هر کدام از این گزینه‌ها، تنظیمات مربوط به آنها به طور خودکار، اعمل می‌شود.

\section{مطالب پایان‌نامه را چطور بنویسم؟}
\subsection{نوشتن فصل‌ها}
همان‌طور که در بخش 
\ref{sec2}
گفته شد، برای جلوگیری از شلوغی و سردرگمی کاربر در هنگام حروف‌چینی، قسمت‌های مختلف پایان‌نامه از جمله فصل‌ها، در فایل‌های جداگانه‌ای قرار داده شده‌اند. 
بنابراین، اگر می‌خواهید مثلاً مطالب فصل ۱ را تایپ کنید، باید فایل‌های 
\verb;AUTthesis.tex;
و
\verb;chapter1;
را باز کنید و محتویات داخل فایل 
\verb;chapter1;
را پاک کرده و مطالب خود را تایپ کنید. توجه کنید که همان‌طور که قبلاً هم گفته شد، تنها فایل قابل اجرا، فایل 
\verb;AUTthesis.tex;
است. لذا برای دیدن حاصل (خروجی) فایل خود، باید فایل  
\verb;chapter1;
را 
\verb;Save;
کرده و سپس فایل 
\verb;AUTthesis.tex;
را اجرا کنید. یک نکته بدیهی که در اینجا وجود دارد، این است که لازم نیست که فصل‌های پایان‌نامه را به ترتیب تایپ کنید. می‌توانید ابتدا مطالب فصل ۳ را تایپ کنید و سپس مطالب فصل ۱ را تایپ کنید.

نکته بسیار مهمی که در اینجا باید گفته شود این است که سیستم
\lr{\TeX},
محتویات یک فایل تِک را به ترتیب پردازش می‌کند. به عنوان مثال، اگه فایلی، دارای ۴ خط دستور باشد، ابتدا خط ۱، بعد خط ۲، بعد خط ۳ و در آخر، خط ۴ پردازش می‌شود. بنابراین، اگر مثلاً مشغول تایپ مطالب فصل ۳ هستید، بهتر است
که دو دستور
\verb~\chapter{مقدمه}
\section{مقدمه}
انبارها ساختمان یا محوطه‌ای هستند که برای نگهداری کالاها استفاده می‌گردند. صنعت‌گران، واردکنندگان، صادرکنندگان، عمده‌فروشان و گمرک، استفاده‌کنندگان انبارها می‌باشند. انبارها معمولاً در شهرها، شهرک‌های صنعتی و کارخانجات ساخته می‌شوند ولی ممکن است جهت سهولت در دریافت و صدور کالا، در کنار راه‌های اصلی، فرودگاه و یا بنادر نیز ساخته شوند تا کالاها مستقیماً به آنجا وارد و یا خارج گردد.

انبار‌ها از لحاظ کاربری ممکن است انواع گوناگونی داشته باشند. انبار محصول، انبار مواد اولیه، انبار قطعات نیم‌سوخته، انبار قطعات یدکی، انبار ابزار آلات، انبار غلات و مخازن تنها تعداد اندکی از انواع انبارها می‌باشند. با توجه به کاربری و تنوع زیاد انواع انبارها، ماشین آلات، سیستم‌ها و تجهیزات سخت‌افزاری و نرم‌افزاری بسیاری متناسب با نوع کاربری انبارها توسعه یافته‌اند. از آنجا که انبارها در فرآیند تجارت هیچ‌گونه ارزش افزوده‌ای ایجاد نمی‌نمایند، مکان‌یابی، جابجایی و خروج کالاها از انبار بسیار خطیر است و لازم است تا عملیات انبارها با حداقل هزینه و حداکثر بهره‌وری انجام پذیرد.

سیستم اداره انبارها با توجه به نوع کاربری و جامعه مورد استفاده بسیار متفاوت است. در برخی از جوامع، انبارها هنوز به صورت سنتی اداره می‌گردند، حال آنکه برخی از انبارها کاملاً خودکار و مکانیزه می‌باشند؛ بدون اینکه نیاز به نیروی کار انسانی داشته باشند و از طریق سیستم‌های دریافت و انتقال خودکار کالاها و نرم افزارهای لجستیکی\LTRfootnote{\lr{Logistic}} مدیریت می‌شوند.

انبارداری به صورت سنتی، در صورت افزایش مقیاس انبار، مشکلات زیادی را به وجود می‌آورد. در این روش هرگونه مشکل پیش‌آمده برای کارکنان انبار باعث ضرر و زیان مالی می‌شود. از این مشکلات می‌توان به خستگی، چندکاری، محیط پر سروصدا و کاهش تمرکز کارکنان اشاره کرد.
 
 
مشکل پیدا کردن و جمع‌آوری کالاهای مورد نیاز در فروشگاه‌ها و انبارهای بزرگ، همواره دغدغه کارخانه‌ها و شرکت‌های توزیع‌کننده و صاحبان فروشگاه‌ها بوده‌است. این مشکل زمانی حیاتی می‌شود که جایگاه کالاها ثابت نباشد و در طول زمان تغییر کند. در این صورت جمع‌آوری کالا به گلوگاه مجموعه تبدیل می‌شود. در در جامعه کنونی و در شرایط فعلی، این روندها بیشتر به صورت انسانی و بدون کمک رایانه انجام می‌شوند. از موارد مشکل‌ساز روند فعلی، می‌توان به افزایش هزینه‌های چیدمان و نگهداری کالاها اشاره کرد. چون چیدمان نامناسب کالاها و جمع‌آوری آن‌ها در انبارها، به نیروی انسانی بیشتری نیاز دارد.
 
این پروژه، امکانی را برای کارکنان فراهم می‌کند که جایگاه اجناس را به وسیله قطعه‌های سخت افزاری بروزرسانی کنند و همچنین نرم‌افزاری در اختیار آنان قرار می‌دهد تا به وسیله آن، بتوانند بسیار راحت‌تر و سریع‌تر موارد مورد نیاز خود را پیدا کنند. همچنین این پروژه به مشتریان فروشگاه‌های بزرگ و زنجیره‌ای کمک می‌کند که اطلاعات و جایگاه کالاهای موجود در لیست خریدشان را دریافت کنند. علاوه بر این، رابطی برای مدیریت مجموعه فراهم شده‌است تا بتوانند وضعیت فعلی مجموعه را زیر نظر داشته باشند و در صورت نیاز، تغییراتی در سیستم اعمال کنند.
 
در این نوشته، ابتدا به تعریف مسئله و راه حل پیشنهادی و همچنین نیازمندی‌های مسئله پرداخته می‌شود. سپس تکنولوژی‌های نرم‌افزاری و همچنین قطعات سخت‌افزاری انتخاب شده معرفی می‌گردند. در قسمت بعد به شرح مفصل پیاده‌سازی راه‌حل پرداخته می‌شود و در نهایت به جمع‌بندی کارهای انجام شده و همچنین کارهایی که در آینده برای این پروژه می‌توان انجام داد پرداخته می‌شود.~
و
\verb~\chapter{تعریف مسئله، راه حل پیشنهادی و تعریف نیازمندی‌ها}
\section{مقدمه}
در این قسمت ابتدا به تعریف و طرح مسئله پرداخته می‌شود. سپس راه‌حلی پیشنهاد داده می‌شود و در ادامه نیازمندی‌های پروژه بر اساس راه حل پیشنهادی تعریف می‌شود.
پیش از آغاز پیاده‌سازی نرم‌افزار لازم است درک جامع و کاملی از مسئله پیدا کنیم و فهرست دقیقی از نیازمندی‌های آن تهیه نماییم. در این فصل سعی شده است پس از بیان مسئله، نیازمندی‌ها و روش حل آن مشخص شوند.

\section{تعریف مسئله}
مدیریت و یافتن کالاها در فروشگاه‌ها و انبار‌های بزرگ، یکی از دغدغه‌های اصلی فروشندگان و انباردارهاست. هنگامی که سفارشی ثبت می‌شود، جمع‌آورنگان محصولات در انبارها وظیفه دارند در انبوهی از کالاها و بخش‌های انبار، کالاهای ثبت‌شده در سفارش را جمع‌آوری کنند. اگر انبار مربوطه تعداد زیادی کالا داشته باشد و یا انباری مربوط به کسب‌و‌کارهایی با محصولات پرمصرف باشد، این موضوع به گلوگاه تبدیل می‌شود. همچنین مشتریان فروشگاه‌های زنجیره‌ای همواره با پیدا کردن لیست محصولات مورد نیاز خود در بین انبوهی از کالاها مشکل دارند. کاهش زمان صرف‌شده برای جمع‌آوری کالاها، برای مدیران 
انبارها، از اهمیت ویژه‌ای برخوردار است.


\section{راه حل پیشنهادی} 
راه حل پیشنهادی برای حل این مسئله، طراحی و پیاده‌سازی سامانه‌ای جهت مدیریت، ایجاد تغییرات و کنترل کالاها در جایگاه‌های متفاوت در این محیط‌هاست. این سامانه متشکل از قطعه‌های سخت‌افزاری و همچنین نرم‌افزار لازم برای تعامل با کاربر است. این سامانه از طریق بوردها و صفحه‌های نمایش لمسی قرار گرفته بر روی آن، در قسمت‌ها و دسته‌های مختلف انبار و فروشگاه، این اجازه را به کارکنان انبارها و فروشگاه‌ها می‌دهد که موجودی کالا‌های خود را به روز‌رسانی کنند و از طریق نرم‌افزار مربوطه، مکان، موجودی و سایر مشخصات کالا را جست‌و‌جو کنند. همچنین به یک برنامه سمت کاربر جهت مدیریت داده‌ها و کاربران نیاز است.


\section{نیازمندی‌های پروژه}
با توجه به موارد مطرح شده در قسمت تعریف مسئله و همچنین در قسمت راه حل پیشنهادی، نیازمندی‌هایی به وجود آمد که در ادامه به آن‌ها پرداخته می‌شود.

\subsection{رابط کاربری مدیریت}
پروژه شامل کاربرها و محصولات، بخش‌ها\LTRfootnote{\lr{Segments}}، سفارش‌ها و دسته‌بندی‌های مختلف است. برای مدیریت و کنترل این قسمت‌ها، به یک محیطی نیاز است که مدیر، دسترسی تغییر بخش‌های مختلف را بر اساس نیاز داشته باشد.

مدیر در محیط مدیریت\LTRfootnote{\lr{Admin Panel}} باید بتواند اعمال زیر را انجام دهد:
\begin{itemize}
   	\item اضافه کردن، ویرایش و یا حذف یک محصول
   	\item اضافه کردن، ویرایش و یا حذف یک دسته‌بندی
   	\item اضافه کردن، ویرایش و یا حذف یک بخش
   	\item مشاهده و ویرایش سفارش‌های ثبت شده
\end{itemize}
با توجه به این که ممکن است بر اساس نیاز، لازم شود افراد دیگری غیر از مدیر اصلی توانایی تغییر بخش‌های ذکر شده را داشته باشند، باید امکانِ دادن دسترسی به کاربران دیگر نیز داده شود. همچنین نیاز است سطح دسترسی کاربران تحت عنوان نقش‌های مختلف نیز تعریف شوند. از وظایف مدیر اصلی رابط کاربری مدیریت، تعیین دسترسی‌های کاربران و نقش‌های مختلف برنامه است.

\subsection{برنامه جمع‌آوری محصولات سفارش‌ها}
جمع‌آورندگان\LTRfootnote{\lr{Picker}} داخل انبار و فروشگاه نیاز دارند تا از سفارش‌های ثبت شده مطلع شوند تا به جمع‌آوری محصولات آن سفارش بپردازند. همچنین پس از جمع‌آوری محصولات، به مدیر و بقیه جمع‌آورندگان باید اطلاع دهند که سفارش جمع‌آوری گردید.

این برنامه شامل موارد زیر است:
\begin{itemize}
	\item مشاهده سفارش‌های ثبت شده و وضعیت آن‌ها
	\item مشاهده محصولات سفارش
	\item اطلاع‌ دادن پس از جمع‌آوری محصولات سفارش
\end{itemize}

\subsection{برنامه انتخاب محصولات بخش‌ها‌}
بخش‌های مختلف در انبارها و فروشگاه‌ها دارای محصولات متفاوتی هستند. این محصولات در طول زمان اضافه، کم ویا جابجا می‌شوند.

در این پروژه به قطعات سخت‌افزاری جهت قرارگیری بر روی بخش‌های مختلف نیاز است تا کاربر بتواند به راحتی با آن تعامل برقرار کند و محصولات فعلی بخش  و موجودی آن‌ها را ثبت کنند. برای ثبت محصولات، علاوه بر قطعات سخت‌افزاری، به برنامه‌ای نیاز است تا محصولات موجود را نمایش دهد و پس از تغییر جایگاه محصولات، آن تغییر را در پایگاه‌داده ثبت کند.

\section{نتیجه‌گیری}
در این قسمت مسئله مورد نظر تعریف و شرح داده شد. همچنین راه‌حلی پیشنهادی برای مسئله ارائه شد. در نهایت و با توجه به راه حل، نیازمندی‌های مسئله تعریف شد. این نیازمندی‌ها شامل برنامه‌های سمت کاربر و همچنین مدیریت مجموعه است.
~
را در فایل 
\verb~AUTthesis.tex~،
غیرفعال%
\RTLfootnote{
برای غیرفعال کردن یک دستور، کافی است پشت آن، یک علامت
\%
 بگذارید.
}
 کنید. زیرا در غیر این صورت، ابتدا مطالب فصل ۱ و ۲ پردازش شده (که به درد ما نمی‌خورد؛ چون ما می‌خواهیم خروجی فصل ۳ را ببینیم) و سپس مطالب فصل ۳ پردازش می‌شود و این کار باعث طولانی شدن زمان اجرا می‌شود. زیرا هر چقدر حجم فایل اجرا شده، بیشتر باشد، زمان بیشتری هم برای اجرای آن، صرف می‌شود.

\subsection{مراجع}
برای وارد کردن مراجع به فصل 2
مراجعه کنید.
\subsection{واژه‌نامه فارسی به انگلیسی و برعکس}
برای وارد کردن واژه‌نامه فارسی به انگلیسی و برعکس، بهتر است مانند روش بکار رفته در فایل‌های 
\verb;dicfa2en;
و
\verb;dicen2fa;
عمل کنید.

\section{اگر سوالی داشتم، از کی بپرسم؟}
برای پرسیدن سوال‌های خود در مورد حروف‌چینی با زی‌پرشین،  می‌توانید به
 \href{http://forum.parsilatex.com}{تالار گفتگوی پارسی‌لاتک}%
\LTRfootnote{\url{http://www.forum.parsilatex.com}}
مراجعه کنید. شما هم می‌توانید روزی به سوال‌های دیگران در این تالار، جواب بدهید.
