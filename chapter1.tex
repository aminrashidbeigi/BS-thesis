\chapter{مقدمه}
\section{مقدمه}
<<<<<<< HEAD
انبارها ساختمان یا محوطه‌ای هستند که برای نگهداری کالاها استفاده می‌گردند. صنعت‌گران، واردکنندگان، صادرکنندگان، عمده‌فروشان و گمرک، استفاده‌کنندگان انبارها می‌باشند. انبارها معمولاً در شهرها، شهرک‌های صنعتی و کارخانجات ساخته می‌شوند ولی ممکن است جهت سهولت در دریافت و صدور کالا، در کنار راه‌های اصلی، فرودگاه و یا بنادر نیز ساخته شوند تا کالاها مستقیماً به آنجا وارد و یا خارج گردد.

انبار‌ها از لحاظ کاربری ممکن است انواع گوناگونی داشته باشند. انبار محصول، انبار مواد اولیه، انبار قطعات نیم‌سوخته، انبار قطعات یدکی، انبار ابزار آلات، انبار غلات و مخازن تنها تعداد اندکی از انواع انبارها می‌باشند. با توجه به کاربری و تنوع زیاد انواع انبارها، ماشین آلات، سیستم‌ها و تجهیزات سخت‌افزاری و نرم‌افزاری بسیاری متناسب با نوع کاربری انبارها توسعه یافته‌اند. از آنجا که انبارها در فرآیند تجارت هیچ‌گونه ارزش افزوده‌ای ایجاد نمی‌نمایند، مکان‌یابی، جابجایی و خروج کالاها از انبار بسیار خطیر است و لازم است تا عملیات انبارها با حداقل هزینه و حداکثر بهره‌وری انجام پذیرد.
=======
انبارها ساختمان یا محوطه‌ای هستند که برای نگهداری کالاها استفاده می‌گردند.صنعت‌گران، واردکنندگان، صادرکنندگان، عمده‌فروشان و گمرک، استفاده‌کنندگان انبارها می‌باشند. انبارها معمولاً در شهرها، شهرک‌های صنعتی و کارخانجات ساخته می‌شوند ولی ممکن است جهت سهولت در دریافت و صدور کالا، در کنار راه‌های اصلی، فرودگاه و یا بنادر نیز ساخته شوند تا کالاها مستقیماً به آنجا وارد و یا خارج گردد.

انبار‌ها از لحاظ کاربری ممکن است انواع گوناگونی داشته باشند. انبار محصول، انبار مواد اولیه، انبار قطعات نیم‌سوخته، انبار قطعات یدکی، انبار ابزار آلات، انبار غلات و مخازن تنها تعداد اندکی از انواع انبارها می‌باشند. با توجه به کاربری و تنوع زیاد انواع انبارها، ماشین آلات، سیستم‌ها و تجهیزات سخت‌افزاری و نرم‌افزاری بسیاری متناسب با نوع کاربری انبارها توسعه یافته‌اند. از آنجا که انبارها در فرآیند تجارت هیچ‌گونه ارزش افزوده‌ای ایجاد نمی‌نمایند، مکان‌یابی، جابجایی و خروج کالاها از انبار بسیار خطیر می‌باشد و لازم است تا عملیات انبارها با حداقل هزینه و حداکثر بهره‌وری انجام پذیرد.
>>>>>>> 0b914906bc0a1f3ca7b01ffa78967759bf4780dc

سیستم اداره انبارها با توجه به نوع کاربری و جامعه مورد استفاده بسیار متفاوت است. در برخی از جوامع، انبارها هنوز به صورت سنتی اداره می‌گردند، حال آنکه برخی از انبارها کاملاً خودکار و مکانیزه می‌باشند؛ بدون اینکه نیاز به نیروی کار انسانی داشته باشند و از طریق سیستم‌های دریافت و انتقال خودکار کالاها و نرم افزارهای لجستیکی\LTRfootnote{\lr{Logistic}} مدیریت می‌شوند.

انبارداری به صورت سنتی، در صورت افزایش مقیاس انبار، مشکلات زیادی را به وجود می‌آورد. در این روش هرگونه مشکل پیش‌آمده برای کارکنان انبار باعث ضرر و زیان مالی می‌شود. از این مشکلات می‌توان به خستگی، چندکاری، محیط پر سروصدا و کاهش تمرکز کارکنان اشاره کرد.
 
 
<<<<<<< HEAD
مشکل پیدا کردن و جمع‌آوری کالاهای مورد نیاز در فروشگاه‌ها و انبارهای بزرگ، همواره دغدغه کارخانه‌ها و شرکت‌های توزیع‌کننده و صاحبان فروشگاه‌ها بوده‌است. این مشکل زمانی حیاتی می‌شود که جایگاه کالاها ثابت نباشد و در طول زمان تغییر کند. در این صورت جمع‌آوری کالا به گلوگاه مجموعه تبدیل می‌شود. در در جامعه کنونی و در شرایط فعلی، این روندها بیشتر به صورت انسانی و بدون کمک رایانه انجام می‌شوند. از موارد مشکل‌ساز روند فعلی، می‌توان به افزایش هزینه‌های چیدمان و نگهداری کالاها اشاره کرد. چون چیدمان نامناسب کالاها و جمع‌آوری آن‌ها در انبارها، به نیروی انسانی بیشتری نیاز دارد.
=======
مشکل پیدا کردن و جمع‌آوری کالاهای مورد نیاز در فروشگاه‌ها و انبارهای بزرگ، همواره دغدغه کارخانه‌ها و شرکت‌های توزیع‌کننده و صاحبان فروشگاه‌ها بوده‌است. این مشکل زمانی حیاتی می‌شود که جایگاه کالاها ثابت نباشد و در طول زمان تغییر کند. در این صورت جمع‌آوری کالا به گلوگاه مجموعه تبدیل می‌شود. در در جامعه کنونی و در شرایط فعلی، این روندها بیشتر به صورت انسانی و بدون کمک رایانه انجام می‌شوند. از موارد مشکل‌ساز روند فعلی، می‌توان به افزایش هزینه‌های چیدمان و نگه‌داری کالاها اشاره کرد. چون چیدمان نامناسب کالاها و جمع‌آوری آن‌ها در انبارها، به نیروی انسانی بیشتری نیاز دارد.
>>>>>>> 0b914906bc0a1f3ca7b01ffa78967759bf4780dc
 
این پروژه، امکانی را برای کارکنان فراهم می‌کند که جایگاه اجناس را به وسیله قطعه‌های سخت افزاری بروزرسانی کنند و همچنین نرم‌افزاری در اختیار آنان قرار می‌دهد تا به وسیله آن، بتوانند بسیار راحت‌تر و سریع‌تر موارد مورد نیاز خود را پیدا کنند. همچنین این پروژه به مشتریان فروشگاه‌های بزرگ و زنجیره‌ای کمک می‌کند که اطلاعات و جایگاه کالاهای موجود در لیست خریدشان را دریافت کنند. علاوه بر این، رابطی برای مدیریت مجموعه فراهم شده‌است تا بتوانند وضعیت فعلی مجموعه را زیر نظر داشته باشند و در صورت نیاز، تغییراتی در سیستم اعمال کنند.
 
در این نوشته، ابتدا به تعریف مسئله و راه حل پیشنهادی و همچنین نیازمندی‌های مسئله پرداخته می‌شود. سپس تکنولوژی‌های نرم‌افزاری و همچنین قطعات سخت‌افزاری انتخاب شده معرفی می‌گردند. در قسمت بعد به شرح مفصل پیاده‌سازی راه‌حل پرداخته می‌شود و در نهایت به جمع‌بندی کارهای انجام شده و همچنین کارهایی که در آینده برای این پروژه می‌توان انجام داد پرداخته می‌شود.